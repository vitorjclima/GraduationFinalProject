\begin{resumo}
Plataformas e \textit{software} estão sendo disponibilizados como serviços em ambientes de Computação em Nuvem com um objetivo, a escalabilidade. Um sistema é descrito como escalável se permanece eficiente quando há um aumento significativo no número de recursos e no número de usuários. Para alcançar escalabilidade, é preciso evitar gargalos no sistema. O objetivo deste trabalho foi estudar modelos de escalabilidade, a partir de conceitos de Computação em Nuvem. Isto inclui realizar estudos sobre ferramentas de virtualização, balanceamento de carga, servidores web, sistema de arquivos distribuídos, entre outras. Foram estudadas diferentes métricas de desempenho para avaliar os modelos, tais como: tempo de resposta, capacidade e eficiência do sistema.
 
 \vspace{\onelineskip}
    
 \noindent
 \textbf{Palavras-chaves}: computação em nuvem, escalabilidade, balanceamento de carga, virtualização.
\end{resumo}