Este capítulo descreve a complexidade da implantação de projetos distribuídos ao longo deste trabalho. Devido a esta complexidade foi necessário fazer uma revisão no objetivo inicial do trabalho que era realizar o estudo dos modelos de escalabilidade em Computação em Nuvem para o projeto Maritaca, porém acabou por ficar restrito à escalabilidade, elasticidade e robustez do banco de dados distribuído (Cassandra).


O projeto Maritaca (\textbf{MAR}itaca \textbf{I}s a \textbf{T}ool to cre\textbf{A}te \textbf{C}ellular phone \textbf{A}pplications) busca suprir as necessidades de automatizar o processo de coleta de dados móveis (CMD) provendo uma infraestrutura baseada na nuvem para criação de aplicativos móveis para \textit{smartphones} Android (\textit{apks}\footnote{Arquivos executáveis Android.}). Esses aplicativos representam os formulários digitais. O Maritaca proporciona uma \textit{interface web} interativa para a concepção do formulário e o gerenciamento das informações coletadas. O projeto possui três vertentes principais: a aplicação móvel Android, os serviços web e os repositórios de dados. 


O estudo dos modelos dependia preliminarmente do funcionamento pleno do sistema Maritaca, o que não ocorreu. Para automatizar o processo de download, instalação e configuração de pacotes essenciais para o funcionamento do sistema, foi necessário alterar o foco inicial deste trabalho para a criação de \textit{scripts} que realizassem o trabalho inicial. Porém, durante a criação destes, o projeto Maritaca demonstrou incompatibilidade de versões dos pacotes e/ou documentação deficiente do projeto com diversas distribuições Linux.

Primeiramente, foi decidido testar o processo de instalação no sistema operacional Ubuntu Server 14.04.2. Em função de diversos problemas, tentou-se utilizar outros sistemas operacionais, tais como OpenSuse 12, CentOS 7 e Fedora 21.

No OpenSuse foi detectado problemas na resolução de dependências dos pacotes utilizados pelo Maritaca. Durante a instalação dos pacotes, os mesmos apresentavam falhas em função do sistema operacional não resolver as dependências exigidas pelos pacotes.

No sistema CentOS, os pacotes exigidos pelo Maritaca foram instalados com sucesso. Porém, ao iniciar o projeto compilado, este retornara erro 404, motivo que não se conseguiu resolver.

Após o teste em diversos sistemas operacionais, descobriu-se que a versão do Maven utilizada para compilação do projeto estava com problemas na indexação dos pacotes utilizados pelo projeto. Em função disto, optou-se por alterar a versão do Maven, de 2 para 3, e reiniciou-se a criação dos scripts no sistema operacional Ubuntu Server 14.04.2.

Ainda assim, o TomCat não conseguia iniciar o projeto compilado. Por isto, iniciou-se a depuração do log do TomCat, onde detectou-se ser fundamental a criação da pasta hadoop\_mounted dentro na pasta var do sistema operacional. Porém novamente o TomCat apresentou falhas ao iniciar o projeto.

Desta forma, optou-se pela utilização do sistema operacional Fedora 21. Durante a instalação do Maven 3, verificou-se que o mesmo exigia a versão 1.8 do Java Open JDK, e ao compilar o projeto com esta versão do Java, novos problemas surgiram. Sendo assim, retomou-se a criação dos \textit{scripts} utilizando o Ubuntu Server 14.04.2.

O projeto foi então recompilado utilizando-se a versão 1.6 do Java, e iniciou-se o servidor TomCat utilizando o novo arquivo compilado, desta vez, o projeto executou com sucesso.

Ao testar a geração de aplicativos com o auxílio do Maritaca, este não conseguia gerar os \textit{apks}. Ao realizar nova depuração do log do TomCat, detectou-se a ausência de pacotes fundamentais do Android SDK para a geração dos apks. Foram então inseridos no Android SDK um conjunto de pacotes obsoletos. Testou-se novamente a geração dos apks e novo erro surgiu no log do TomCat, indicando que outras dependências do Android SDK não estavam instaladas. Instalou-se as dependências exigidas (\textit{lib32z1, lib32ncurses5, lib32bz2-1.0 e lib32stdc++6}). Com isto, a geração dos \textit{apks} pelo Maritaca passou a funcionar com sucesso.

