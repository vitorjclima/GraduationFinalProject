Computação em Nuvem é atualmente um dos principais tópicos em sistemas distribuídos. As novas formas de oferecer serviços de software tem atraído as empresas e a sociedade. Segundo \citeonline{JISA2010}, isto se deve a vários fatores, como:

\begin{itemize}
    \item baixo investimento inicial;
    \item redução de custos operacionais;
    \item escalabilidade;
    \item redução de riscos e despesas de manutenção.
\end{itemize}

Num estudo realizado em 2013, os analistas da Frost \& Sullivan, empresa global de consultoria especializada em estratégia de crescimento, concluíram que o mercado brasileiro de computação esta mais adepto a Computação em Nuvem que os demais mercados do mundo. Dados extraídos da pesquisa realizada em São Paulo mostram que 39\% das companhias que atuam na América Latina possuem aplicativos hospedados na nuvem \cite{Silva2014}.

Desta forma, saber avaliar e definir qual estratégia de escalabilidade a ser utilizada, tem sido cada vez mais importante na Computação em Nuvem. Tais estratégias representam ganhos significativos para a expansão dos serviços e recursos oferecidos \cite{JISA2011}.

Pode-se citar a Amazon Web Services (AWS) como exemplo de empresa que investiu na Computação em Nuvem desde 2006 e hoje este é o seu principal serviço. Esta empresa desenvolveu uma plataforma que facilita e viabiliza o uso da tecnologia em nuvem, fazendo com que milhares de empresas migrassem seus serviços para os seus servidores, como por exemplo: Novartis, Nokia, Pfizer, Adobe, Netflix, entre outras \cite{AmazonWebServices2015}. Conforme estudo realizado pela Netflix e citado na seção \ref{sec:cassjmeter}, os testes realizados na plataforma Amazon revelam que os serviços da Netflix conseguiram atender a um maior número de clientes a medida que se aumenta a escalabilidade da nuvem. 

A utilização da virtualização nestes sistemas permite que um hardware seja utilizado para executar vários sistemas operacionais ao mesmo tempo, sejam eles sistemas iguais ou diferentes, visando o máximo aproveitamento dos recursos computacionais. Com isso, pode-se ter diversas redes privadas dentro de um mesmo servidor. Para garantir segurança, desempenho e confiabilidade destas redes são utilizados diversos tipos de isolamento e de configurações.

Dentro destas redes e sistemas virtuais, os dados são geralmente armazenados em um banco de dados NoSQL, distribuído e rápido, com o qual é possível trabalhar com grande volume de dados. Desta forma, o banco de dados escolhido para ser estudado neste trabalho foi o Cassandra. Ele é um banco de dados baseado no tipo chave/valor, onde os dados são identificados através de uma chave e cujas principais características são: escalabilidade, disponibilidade, tolerância a falhas e tempo de resposta rápido \cite{Silva}.

Sendo assim, neste trabalho, foram investigados produtos utilizados para alcançar escalabilidade e os ambientes computacionais ditos escaláveis. Também foram realizadas simulações de escalabilidade no banco de dados para os serviços oferecidos pelo projeto Maritaca \cite{maritaca}. Nos quais foram avaliados o comportamento do sistema em relação ao volume de dados e tráfego, quando estes aumentam ou diminuem, e à detecção de gargalos. Para tais avaliações, utilizou-se o JMeter e o CassJMeter para simular o funcionamento do sistema a ser avaliado e realizar os testes na base Cassandra.

Para isto, este trabalho foi organizado da seguinte forma: o capítulo \ref{cap:fundTeorica} aborda conceitos de Computação em Nuvem, diferenciando os tipos de serviços e principais estruturas. O capítulo \ref{cap:ferramentas} descreve as ferramentas utilizadas neste trabalho. O capítulo \ref{cap:ambExperimental} detalha o servidor utilizado para criação dos ambientes experimentais e os modelos de escalabilidade, bem como alguns diagramas que exemplificam conceitos como: redes virtuais, redes privadas, isolamento de redes, balanceamento de carga e o relacionamento entre estes. O detalhamento de bases Cassandra operando em \textit{cluster} e o plano de testes das mesmas estão descritos no capítulo \ref{cap:cassandra}. A seleção de técnicas e métricas para análise de desempenho, bem como as métricas mais usadas são apresentadas no capítulo \ref{cap:analisePerformance}. No capítulo \ref{cap:resultados} são tratados os assuntos relacionados aos resultados obtidos. No capítulo \ref{cap:dificuldadesEncontradas}, são apresentadas as principais dificuldades encontradas na execução deste trabalho e no capítulo \ref{cap:conclusoes} é feito um fechamento recapitulando pontos importantes discutidos ao longo do trabalho e as conclusões obtidas.